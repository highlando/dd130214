\documentclass{beamer}
\usepackage{beamerthemeshadow}
%\usepackage{natbib}
\usepackage{amssymb,amsmath,amsthm,mathtools,amscd}
%\beamersetuncovermixins{\opaqueness<1>{25}}{\opaqueness<2->{15}}
\usepackage{float}
\usepackage{tikz}
\usetikzlibrary{matrix}

\useinnertheme{rounded}
\usecolortheme{beetle}
\setbeamercolor{normal text}{bg=white!20} 
\setbeamercolor{title}{bg=blue!10!gray}
\setbeamercolor{block title}{fg=black,bg=blue!10!gray!50!}

\setbeamercolor{section in toc}{fg=black}
%\logo{\includegraphics[height=0.8cm]{TU-LogoSchrift2line.jpg}} 
\setbeamercolor{logo}{bg=white}
\setbeamercolor{itemize item}{fg=gray}

\DeclareMathOperator{\re}{Re}

\newenvironment{recap}{\begin{small}\color{gray}\begin{flushright}}{\end{flushright}\end{small}}
\input{mathComAbb}


\begin{document}
%%% Spaces
\def\Hdf{H_{{df}}}
\def\sdf{_{\text{df}}}
\def\sg{_{\text{c}}}
\def\Hg{H_{{g}}}
\def\sg{_{g}}
\def\nv{n_v}
\def\np{n_p}


%%% Projectors
% \def\Prj{P_{[V\sdf|V\sg]}}
\def\Prj{P\sdf}
\def\PHdf{\mathcal P_{\Hdf}}
\def\PHpdf{\mathcal P_{\Hdf'}}

\title[Decoupling of semi-explicit index-2 DAEs]{ Decoupling of Differential-Algebraic Equations \\ with Application in Flow Control}


\author{Jan Heiland}
\institute{ Absolventen-Seminar Numerische Mathematik }
\date{\today} 

\titlegraphic{\includegraphics[scale=0.009]{TUBerlin_Logo_rot.jpg} }

\frame{\titlepage} 

\frame{
\vfill
\tableofcontents
} 

\section{Decoupling of the DAEs}
\subsection{Motivating Example}

\frame{
\frametitle{Semi-discrete Navier-Stokes Equation}
For $M$, $N$ in $\mathbb R^{\nv, \nv}$, $J \in \mathbb R^{\np}$, with $n_v$, $n_p\in \mathbb N$, and for $v\in \mathcal C^1(0,T;\mathbb R^{n_v})$ and $p\in \mathcal C(0,T;\mathbb R^{n_v})$ consider
\begin{align*}
M \dot v - Av - J^Tp &= f, \\
Jv &= 0 \quad \text{in } (0,T) \times \mathbb R^{n_v}.
\end{align*}
Assume that $M$ are $JM^{-1}J^T$ are invertible.
}

\frame{
\frametitle{Preparing for the decoupling}
\begin{recap}
$M \dot v - Av - J^Tp = f,$ \\
$Jv = 0$ \\ 
% $M$, $JM^{-1}J^T$ invertible
\end{recap}
We observe:
\begin{itemize}
	\item If $Jv=0$ for all time, then $J\dot v = 0$ for all time,
	\item and $\mathbb R^{\nv}=\ker J \oplus \image M^{-1}J^T. $
\end{itemize}
and we define:
\begin{itemize}
	\item the spaces $V\sdf := \ker J$ and $V\sg := \image M^{-1}J^T $
	\item and the projector $\Prj\colon \mathbb R^{\nv}\to \mathbb R^{\nv}$, 
	\item with $\image \Prj =V\sdf $ and $\ker \Prj = V\sg$.
\end{itemize}

}

\frame{
\frametitle{Decoupling}
\begin{recap}
$M \dot v - Av - J^Tp = f,$ \\
$Jv = 0$ \\ 
$J\dot v = 0$\\
$\mathbb R^{\nv}=\ker J \oplus \image M^{-1}J^T = V\sdf \oplus V\sg$
\end{recap}

Considering the parts of $$\dot v(t) - M^{-1}Av(t) - M^{-1}J^Tp(t) = M^{-1}f(t)\in \mathbb R^{\nv}$$ in $V\sdf$ and $V\sg$, we obtain two equations:
\begin{equation*}
	\dot v - \Prj M^{-1}Av = \Prj M^{-1}f
\end{equation*} 
as the part in $V\sdf$ and, having applied $J$ to the remainder part,
\begin{equation*}
	-JM^{-1}Av-JM{-1}J^T p = JM^{-1}f,
\end{equation*}
which is also known as the Pressure Poisson Equation.

}
\subsection{Infinite-dimensional Issues}
\frame{
\frametitle{The Abstract Setup}
\begin{itemize}
	\item Banach space $V \hookrightarrow H$ Hilbert space (dense and continuously embedded)
	\item Riesz isomorphism $j'\colon H \to H'$ identifies $H$ with its dual $H'$
	\item Then, $V\hookrightarrow H \cong H' \hookrightarrow V'$
\end{itemize}
\vfill
 We look for $v\in \bigl( (0,T) \to V \bigr)$ with $\dot v(t) \in V'$ and for $p \in \bigl( (0,T) \to Q_H \bigr)$ that satisfy
	\begin{align*}
		\dot v -Av-J'p&=f \quad\text{ in }(0,T) \times V', \\
		Jv&=0 \quad\text{ in }(0,T) \times Q_H'
	\end{align*}
with $V\hookrightarrow H \hookrightarrow V'$ as defined above, and with a Hilbert space $Q_H$.
}
\frame{
\frametitle{The Abstract NSE}
	\begin{recap}
		$v(t)\in V$, $\dot v(t) \in V'$, $p(t)\in Q_H$, \\
		$\dot v -Av-J'p=f$, \\
		$Jv=0$
	\end{recap}
	For the Navier-Stokes equations on a domain $\Omega \subset \mathbb R^{d}$, $d=\{2,3\}$, the spaces are given as 
\begin{itemize}
	\item $V:=[W^{1,2}(\Omega)]^d$, $H:=[L^2(\Omega)]^d$, and $Q_H:= L^2(\Omega)/\mathbb R^{}$ 
\end{itemize}
and the operators as 
\begin{itemize}
	\item $A:= \triangle \colon V\to V':=[W^{-1,2}]^d$
	\item $J:= \dive \colon V\to Q_H':=(L^2/\mathbb R^{})'  $
	\item $J':= \nabla \colon Q_H'\to V'$
\end{itemize}
}

\frame{
\begin{recap}
$J\dot v = 0$\\
$\mathbb R^{\nv}=\ker J \oplus \image M^{-1}J^T$ 
\end{recap}
\frametitle{Decoupling - what goes wrong}
We want to decouple 
$$\dot v - \mathcal Av - \mathcal J'p = f$$
on $(0,T) \times V'$, with $v(t) \in V$ in the same way as the finite dimensional equations. However the basic assumptions fail, because of one major reason: $$V \subsetneq V'.$$
This means, in particular, that
\begin{itemize}
	\item $\mathcal J\dot v = ?\quad $ (not defined yet)
	\item $V = \ker \mathcal J \oplus ? \quad $($\image \mathcal J' \in V'$)
\end{itemize}
\vfill
}
\section{Decoupling of the ADAEs}
\frame{
\frametitle{ Two problems, one solution: Regularity }
\begin{recap}
	$\mathcal J\dot v = ?$ \\
$V = \ker \mathcal J \oplus ?$
\end{recap}
We had $Q_H \cong Q_H'$ and  $V\hookrightarrow H \cong H' \hookrightarrow V'$, \\ with operators $\mathcal J$ and $\mathcal J'$ as 
\begin{figure}[htb]
	\begin{tikzpicture}
		\matrix (m) [matrix of math nodes, row sep=2em,
		column sep=0.01em]{
		V\pgfmatrixnextcell  \hookrightarrow \pgfmatrixnextcell  H \pgfmatrixnextcell  \cong\pgfmatrixnextcell  H'\pgfmatrixnextcell  \hookrightarrow \pgfmatrixnextcell V' \\
		~ \pgfmatrixnextcell  \quad ~ \quad \pgfmatrixnextcell  Q_H \pgfmatrixnextcell  \cong\pgfmatrixnextcell  Q_H '\pgfmatrixnextcell  \quad ~ \pgfmatrixnextcell ~  \\
		};
		\path[-stealth]
		%(m-1-1) edge [densely dotted] (m-2-5) 
		(m-1-1) edge  node [left, anchor=north, pos=0.1] {$ \mathcal J $} (m-2-5) 
		(m-2-3) edge  node [right, anchor=north, pos=0.9] {$\mathcal J'$} (m-1-7);
	\end{tikzpicture}
	\label{fig:shiftJ1J2}
\end{figure}
we introduce a Banach space $Q\hookrightarrow Q_H$, such that $\mathcal J'(Q) \subset H'$
}

\frame{
\frametitle{ Two problems, one solution: Regularity }
\begin{recap}
	$\mathcal J\dot v = ?$ \\
$V = \ker \mathcal J \oplus ?$
\end{recap}
We had $Q_H \cong Q_H'$ and  $V\subset H \cong H' \subset V'$, \\ with operators $\mathcal J$ and $\mathcal J'$ as 
\begin{figure}[htb]
	\begin{tikzpicture}
		\matrix (m) [matrix of math nodes, row sep=2em,
		column sep=0.01em]{
		V\pgfmatrixnextcell  \hookrightarrow \pgfmatrixnextcell  H \pgfmatrixnextcell  \cong\pgfmatrixnextcell  H'\pgfmatrixnextcell  \hookrightarrow \pgfmatrixnextcell V' \\
		Q \pgfmatrixnextcell  \quad \hookrightarrow \quad \pgfmatrixnextcell  Q_H \pgfmatrixnextcell  \cong\pgfmatrixnextcell  Q_H '\pgfmatrixnextcell  \quad \hookrightarrow \pgfmatrixnextcell Q'  \\
		};
		\path[-stealth]
		%(m-1-1) edge [densely dotted] (m-2-5) 
		(m-1-1) edge  node [left, anchor=north, pos=0.1] {$ \mathcal J $} (m-2-5) 
		(m-2-3) edge  node [right, anchor=north, pos=0.9] {$\mathcal J'$} (m-1-7);
	\end{tikzpicture}
	\label{fig:shiftJ1J2}
\end{figure}
we introduce a Banach space $Q\hookrightarrow Q_H$, such that $\mathcal J'(Q) \subset H'$
}

\frame{
\frametitle{Smoothness Assumptions}
\begin{recap}
	$\mathcal J'\colon Q_H \to V'$ \\
	$Q\hookrightarrow Q_H, \quad H'\hookrightarrow V'$
\end{recap}

\begin{block}{Assumption [S1]}
	We assume that the shift of $\mathcal J' \colon Q_H \to V'$ ``to the left``:
$$
\bar {\mathcal J'}\colon Q \to H'
$$
is bounded. 
\end{block}

\begin{block}{Assumption [S2]}
	For more regular data $f(t) \in H'$ (rather than in $V'$) any corresponding solution $(v,p)$ is such that $\dot v(t)$ and $\bar{\mathcal J'}p(t)$ is in $H'$, rather than in $V'$. 
\end{block}

}

\frame{
\frametitle{Shift of the Equation to $H'$}
For $f\in L^2(0,T; H')$ and with Assumptions [S1] and [S2] we can write the abstract equation as
$$\dot v - \mathcal Av - \mathcal J'p = f \quad \text{in } (0,T)\times H'$$
and show that
\begin{itemize}
	\item $\bar {\mathcal J}\dot v(t) = 0$ in $Q'$
	\item and $H' = j'(\ker \bar{\mathcal J}) \oplus \image \bar{\mathcal J'}$
\end{itemize}
}

\frame{
\frametitle{Summing Up}
\begin{theorem}
	Consider Equation 
	$$\dot v - \mathcal Av - \mathcal Jp = f \quad \text{in }(0,T) \times V'$$
	and $V\hookrightarrow H$, $Q\hookrightarrow Q_H$ and operators $\mathcal J$, $\bar{\mathcal J}$, $\mathcal A$ as defined above.
	For $f\in L^2(0,T; H')$ and with Assumptions [S1] and [S2], one has the decoupling
\begin{align*}
	\dot v  - \mathcal P\sdf\mathcal  Av &= \mathcal P \sdf f  \quad \text{ in } (0,T) \times j(\ker \bar{\mathcal J'}) \\
	-\bar{\mathcal J} j \mathcal Av-\bar {\mathcal J} j \bar{\mathcal J'} p &= \bar{\mathcal J} jf\quad \text{ in } (0,T) \times Q' ,
\end{align*}
where $j\colon H' \to H$ is the Riesz-isomorphism and $\mathcal P\sdf$ is the projector the realizes $H' = j'(\ker \bar{\mathcal J}) \oplus \image \bar{\mathcal J'}$.
\end{theorem}
}


\frame{
\frametitle{What Else?}
Further issues:
\begin{itemize}
	\item Regularity of $\bigl( t \mapsto v(t),p(t) \bigr)$
	\item Initial conditions and consistency
	\item Spatial discretizations
	\item Optimal control problems
\end{itemize}
}

\frame{
\begin{center}
Thanks to Volker Mehrmann and \\
\vspace{0.1in}
\textbf{thank you for your attention.}\\
\vspace{0.2in}
For suggestions and questions please contact me\\
\vspace{0.1in}
\texttt{heiland@math.tu-berlin.de}
\end{center}
}
\end{document}
