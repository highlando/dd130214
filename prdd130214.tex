\documentclass{beamer}
\usepackage{beamerthemeshadow}
%\usepackage{natbib}
\usepackage{amssymb,amsmath,amsthm,mathtools,amscd}
%\beamersetuncovermixins{\opaqueness<1>{25}}{\opaqueness<2->{15}}
\usepackage{float}
\usepackage{tikz}
\usetikzlibrary{matrix}

\useinnertheme{rounded}
\usecolortheme{beetle}
\setbeamercolor{normal text}{bg=white!20} 
\setbeamercolor{title}{bg=blue!10!gray}
\setbeamercolor{block title}{fg=black,bg=blue!10!gray!50!}

\setbeamercolor{section in toc}{fg=black}
%\logo{\includegraphics[height=0.8cm]{TU-LogoSchrift2line.jpg}} 
\setbeamercolor{logo}{bg=white}
\setbeamercolor{itemize item}{fg=gray}

\DeclareMathOperator{\re}{Re}

\newenvironment{recap}{\begin{small}\color{gray}\begin{flushright}}{\end{flushright}\end{small}}
\input{mathComAbb}


\begin{document}
%%% Spaces
\def\Hdf{H_{{df}}}
\def\sdf{_{\text{df}}}
\def\sg{_{\text{c}}}
\def\Hg{H_{{g}}}
\def\sg{_{g}}
\def\nv{n_v}
\def\np{n_p}


%%% Projectors
% \def\Prj{P_{[V\sdf|V\sg]}}
\def\Prj{P\sdf}
\def\PHdf{\mathcal P_{\Hdf}}
\def\PHpdf{\mathcal P_{\Hdf'}}

\title[Decoupling of semi-explicit index-2 DAEs]{ Decoupling of Differential-Algebraic Equations \\ with Application in Flow Control}


\author{Jan Heiland}
\institute{ Absolventen-Seminar Numerische Mathematik }
\date{\today} 

\titlegraphic{\includegraphics[scale=0.009]{TUBerlin_Logo_rot.jpg} }

\frame{\titlepage} 

\frame{
\vfill
\tableofcontents
} 

\section{Decoupling of the DAEs}
\subsection{Motivating Example}

\frame{
\frametitle{Semi-discrete Navier-Stokes Equation}
For $M$, $N$ in $\mathbb R^{\nv, \nv}$, $J \in \mathbb R^{\np}$, with $n_v$, $n_p\in \mathbb N$, and for $v\in \mathcal C^1(0,T;\mathbb R^{n_v})$ and $p\in \mathcal C(0,T;\mathbb R^{n_v})$ consider
\begin{align*}
M \dot v - Av - J^Tp &= f, \\
Jv &= 0 \quad \text{in } (0,T) \times \mathbb R^{n_v}.
\end{align*}
Assume that $M$ are $JM^{-1}J^T$ are invertible.
}

\frame{
\frametitle{Preparing for the decoupling}
\begin{recap}
$M \dot v - Av - J^Tp = f,$ \\
$Jv = 0$ \\ 
% $M$, $JM^{-1}J^T$ invertible
\end{recap}
We observe:
\begin{itemize}
	\item If $Jv=0$ for all time, then $J\dot v = 0$ for all time,
	\item and $\mathbb R^{\nv}=\ker J \oplus \image M^{-1}J^T. $
\end{itemize}
and we define:
\begin{itemize}
	\item the spaces $V\sdf := \ker J$ and $V\sg := \image M^{-1}J^T $
	\item and the projector $\Prj\colon \mathbb R^{\nv}\to \mathbb R^{\nv}$, 
	\item with $\image \Prj =V\sdf $ and $\ker \Prj = V\sg$.
\end{itemize}

}

\frame{
\frametitle{Decoupling}
\begin{recap}
$M \dot v - Av - J^Tp = f,$ \\
$Jv = 0$ \\ 
$J\dot v = 0$\\
$\mathbb R^{\nv}=\ker J \oplus \image M^{-1}J^T = V\sdf \oplus V\sg$
\end{recap}

Considering the parts of $$\dot v(t) - M^{-1}Av(t) - M^{-1}J^Tp(t) = M^{-1}f(t)\in \mathbb R^{\nv}$$ in $V\sdf$ and $V\sg$, we obtain two equations:
\begin{equation*}
	\dot v - \Prj M^{-1}Av = \Prj M^{-1}f
\end{equation*} 
as the part in $V\sdf$ and, having applied $J$ to the remainder part,
\begin{equation*}
	-JM^{-1}Av-J^{-1}J^T p = J^{-1}f,
\end{equation*}
which is also known as the Pressure Poisson Equation.

}
\subsection{Infinite-dimensional Issues}
\frame{
\frametitle{The Abstract Setup}
\begin{itemize}
	\item Banach space $V \subset H$ Hilbert space (dense and continuously embedded)
	\item Riesz isomorphism $j'\colon H \to H'$ identifies $H$ with its dual $H'$
	\item Then, $V\subset H \cong H' \subset V'$
\end{itemize}
\vfill
 We look for $v\in \bigl( (0,T) \to V \bigr)$ with $\dot v(t) \in V'$ and for $p \in \bigl( (0,T) \to Q_H \bigr)$ that satisfy
	\begin{align*}
		\dot v -Av-J'p&=f \quad\text{ in }(0,T) \times V', \\
		Jv&=0 \quad\text{ in }(0,T) \times Q_H'
	\end{align*}
with $V\subset H \subset V'$ as defined above, and with a Hilbert space $Q_H$.
}
\frame{
\frametitle{The Abstract NSE}
	\begin{recap}
		$v(t)\in V$, $\dot v(t) \in V'$, $p(t)\in Q_H$, \\
		$\dot v -Av-J'p=f$, \\
		$Jv=0$
	\end{recap}
	For the Navier-Stokes equations on a domain $\Omega \subset \mathbb R^{d}$, $d=\{2,3\}$, the spaces are given as 
\begin{itemize}
	\item $V:=[W^{1,2}(\Omega)]^d$, $H:=[L^2(\Omega)]^d$, and $Q_H:= L^2(\Omega)/\mathbb R^{}$ 
\end{itemize}
and the operators as 
\begin{itemize}
	\item $A:= \triangle \colon V\to V':=[W^{-1,2}]^d$
	\item $J:= \dive \colon V\to Q_H':=(L^2/\mathbb R^{})'  $
	\item $J':= \nabla \colon Q_H'\to V'$
\end{itemize}
}

\frame{
\begin{recap}
$J\dot v = 0$\\
$\mathbb R^{\nv}=\ker J \oplus \image M^{-1}J^T$ 
\end{recap}
\frametitle{Decoupling - what goes wrong}
We want to decouple 
$$\dot v - \mathcal Av - \mathcal J'p = f$$
on $(0,T) \times V'$, with $v(t) \in V$ in the same way as the finite dimensional equations. However the basic assumptions fail, because of one major reason: $$V \subsetneq V'.$$
This means, in particular, that
\begin{itemize}
	\item $\mathcal J\dot v = ?\quad $ (not defined yet)
	\item $V = \ker \mathcal J \oplus ? \quad $($\image \mathcal J' \in V'$)
\end{itemize}
\vfill
}
\section{Decoupling of the ADAEs}
\frame{
\frametitle{ Two problems, one solution: Regularity }
\begin{recap}
	$\mathcal J\dot v = ?$ \\
$V = \ker \mathcal J \oplus ?$
\end{recap}
We had $Q_H \cong Q_H'$ and  $V\subset H \cong H' \subset V'$, with operators $\mathcal J$ and $\mathcal J'$ as 
\begin{figure}[htb]
	\begin{tikzpicture}
		\matrix (m) [matrix of math nodes, row sep=2em,
		column sep=0.01em]{
		V\pgfmatrixnextcell  \hookrightarrow \pgfmatrixnextcell  H \pgfmatrixnextcell  \cong\pgfmatrixnextcell  H'\pgfmatrixnextcell  \hookrightarrow \pgfmatrixnextcell V' \\
		~ \pgfmatrixnextcell  \quad ~ \quad \pgfmatrixnextcell  Q_H \pgfmatrixnextcell  \cong\pgfmatrixnextcell  Q_H '\pgfmatrixnextcell  \quad ~ \pgfmatrixnextcell ~  \\
		};
		\path[-stealth]
		%(m-1-1) edge [densely dotted] (m-2-5) 
		(m-1-1) edge  node [left, anchor=north, pos=0.1] {$ \mathcal J $} (m-2-5) 
		(m-2-3) edge  node [right, anchor=north, pos=0.9] {$\mathcal J'$} (m-1-7);
	\end{tikzpicture}
	\label{fig:shiftJ1J2}
\end{figure}
%\begin{figure}[htb]
%	\begin{tikzpicture}
%		\matrix (m) [matrix of math nodes, row sep=4em,
%		column sep=0.1em]{
%		V& \quad \hookrightarrow \quad & H & \cong& H'& \quad \hookrightarrow &V' \\
%		& \quad \hookrightarrow \quad & Q_H & \cong& Q_H '& \quad \hookrightarrow & \\
%		};
%		\path[-stealth]
%		%(m-1-1) edge [densely dotted] (m-2-5) 
%		(m-1-1) edge [densely dotted] node [left, anchor=north, pos=0.1] {$ \mathcal Jht $} (m-2-5) 
%		(m-1-3) edge node [right, anchor=south, pos=0.9] {$\mathcal Jbt$} (m-2-7) 
%		(m-2-1) edge  node [left, anchor=south, pos=0.1] {$\mathcal Jbo'$}(m-1-5)
%		(m-2-3) edge [densely dotted] node [right, anchor=north, pos=0.9] {$\mathcal Jho'$} (m-1-7);
%	\end{tikzpicture}
%\end{figure}

}

\frame{
\frametitle{(Smoothness) Assumption}
\begin{recap}
	$\dot v(t) -A(v(t))-J_1'p(t)=f(t) \quad\text{ in }V'$ \\
	$-J_2v(t)\phantom{-J_1'p(t)}=g(t) \quad\text{ in }Q'$
\end{recap}

\begin{block}{Assumption [S]}
	For more regular data $f(t)\in H'$ (rather than in $V'$) any corresponding solution $(v,p)$ of the abstract DAE invokes $A(v)\in H'$.
\end{block}

\vfill

\begin{corollary}
With $J_1'\colon Q \to H'$, Assumption [S] means that for smoother $f$ the $ADAE$ is posed in the Hilbert space $H'\times Q'$
\end{corollary}
}


\frame{
\frametitle{Split the Spaces}
	By Assumption we have $S:=J_2jJ_1' \colon Q \to Q'$ is invertible, and we can define $L:=J_1'S^{-1}J_2\colon H \to H'$.
\begin{lemma}
\begin{itemize}
		\item $\Hdf:= \ker J_2=\image [I_{H}-jL]$,
		\item $\Hg:= \image jL = \image jJ_1'$,
		\item[and] thus $H=\Hdf\oplus \Hg$,
		\item $\Hg':= \image Lj = j'(\Hg)$,
		\item $\Hdf':= \image [I_{H'}-Lj]=j'(\Hdf)$,
		\item[and] thus $H'=\Hdf'\oplus \Hg'$.
\end{itemize}
\end{lemma}
Define $\PHpdf$, $\PHdf$ -- corresponding projectors onto $\Hdf'$ and $\Hdf$.
}

\frame{
\frametitle{Decouple the Equations}
\begin{recap}
	For smooth $f$ we have \\
	$\dot v(t) -A(v(t))-J_1'p(t)=f(t) \quad\text{ in }H'$ \\
	$-J_2v(t)\phantom{-J_1'p(t)}=g(t) \quad\text{ in }Q'$
\end{recap}
And, having scaled the ADAE by the injective 
 \begin{equation*}
	 \mathcal E_2^{-1}:=\bbmat \PHpdf  & J_1'S^{-1} \\ j_Q'S^{-1}J_2j & -j_Q'S^{-1} \ebmat\colon H'\times Q' \to H'\times Q' . 
 \end{equation*}
(from the projector chain), we get the equivalent system of equation
$$ \PHpdf \dot v  - \PHpdf A(v)-Lv  = \PHpdf f + J_1S^{-1}g \text{ in }H'.$$
plus an equation defining the algebraic variable $p$.
}

\frame{
\frametitle{Split the Equations}
\begin{recap}
$ \PHpdf \dot v  - \PHpdf A(v)-Lv  = \PHpdf f + J_1'S^{-1}g \text{ in }H'$
\end{recap}
Using $H'=\Hdf'\oplus \Hg'$ and $v= \PHdf v+jLv $ we find
that
\begin{equation*}
	jLv = [I-\PHdf]v = -jJ_1S^{-1}g \text{ in }H_g
\end{equation*}
while $\PHdf v$ must solve 
\begin{equation*}
	\dot w  - \PHpdf A(w -jJ_1S^{-1}g)  = \PHpdf f  \text{ in }\Hdf'.
\end{equation*}
}

\frame{
\frametitle{Summing Up}
\begin{theorem}
	Under Assumptions [WP] and [S], for $f$ and $g$ sufficiently smooth,
	any $v$ with values in $V$ solving the ADAE
	\begin{align*}
		\dot v(t) -A(v(t))-J_1'p(t)&=f(t) \quad\text{ in }V'\text{, a.e. in }(0,T), \\
		-J_2v(t)\phantom{-J_1'p(t)}&=g(t) \quad\text{ in }Q'\text{, a.e. in }(0,T).
	\end{align*}
	can be written as $v=\PHdf v + jLv$, where the parts solve 
\begin{align*}
	jLv &= -jJ_1S^{-1}g \text{ in }H_g\\
	\dot \PHdf v  - \PHpdf A(\PHdf v -jJ_1S^{-1}g)  &= \PHpdf f  \text{ in }\Hdf'.
\end{align*}

\end{theorem}
}
\frame{
\begin{center}
Thanks to Volker Mehrmann and \\
\vspace{0.1in}
\textbf{thank you for your attention.}\\
\vspace{0.2in}
For suggestions and questions please contact me\\
\vspace{0.1in}
\texttt{heiland@math.tu-berlin.de}
\end{center}
}

\frame{
\frametitle{References}
\begin{itemize}
	\item \textsc{J.~Heiland} and \textsc{ V.~Mehrmann}, \\ \textit{ Distributed control of linearized Navier-Stokes equations via discretized input/output maps.} \\
		ZAMM - Journal of Applied Mathematics and Mechanics 92(4), 2012 
	\item \textsc{J.~Heiland},  \textsc{V.~Mehrmann},  and  \textsc{M.~Schmidt}, \\ \textit{ A new discretization framework for input/output maps and its application to flow control.} \\
 in: Active Flow Control II, Springer, 2010.
	\item \textsc{D.\,J. Silvester},  \textsc{H.\,C. Elman},  and  \textsc{A.~Ramage}, \\
		\textit{ {IFISS} software package }\\
		\texttt{http://www.manchester.ac.uk/ifiss/} \\
University of Manchester, UK, 2006.
\end{itemize}
}
\frame{ 
\frametitle{References}
\begin{itemize}
	\item Kunkel, Mehrmann, several papers on optimal control for DAEs (e.g. 1997, 2008)
		\vspace{0.1in}
	\item Kurina, M\"arz: \textit {Feedback Solutions of Optimal Control Problems with DAE Constraints} (2007)
		\vspace{0.1in}
	\item Backes: \textit{Optimale Steuerung der linearen DAE im Fall Index 2} (2006)
		\vspace{0.1in}
	\item B\"ansch, Benner: \textit{Stabilization of Incompressible Flow by Riccati-based Feedback} (2010) 
		\vspace{0.1in}
	\item Heinkenschloss, S\"orensen and Sun: \textit{Balanced truncation model reduction for a class of descriptor systems with application to the Oseen equations} (2008)
\end{itemize}
}
\end{document}
